%% ****** Start of file aiptemplate.tex ****** %
%%
%%   This file is part of the files in the distribution of AIP substyles for REVTeX4.
%%   Version 4.1 of 9 October 2009.
%%
%
% This is a template for producing documents for use with 
% the REVTEX 4.1 document class and the AIP substyles.
% 
% Copy this file to another name and then work on that file.
% That way, you always have this original template file to use.

\documentclass[aip,amsmath,amssymb,reprint,onecolumn]{revtex4-1}
%\documentclass[aip,reprint]{revtex4-1}

\usepackage{graphicx,hyperref}
% \usepackage{pdfsync}

\newcommand{\relphantom}[1]{\phantom{\mathrel{#1}}}

\begin{document}

% Use the \preprint command to place your local institutional report number 
% on the title page in preprint mode.
% Multiple \preprint commands are allowed.
%\preprint{}

\title{Quasi-discrete Hankel transform with Neumann boundary conditions} %Title of paper

% repeat the \author .. \affiliation  etc. as needed
% \email, \thanks, \homepage, \altaffiliation all apply to the current author.
% Explanatory text should go in the []'s, 
% actual e-mail address or url should go in the {}'s for \email and \homepage.
% Please use the appropriate macro for the type of information

% \affiliation command applies to all authors since the last \affiliation command. 
% The \affiliation command should follow the other information.

\author{G.R. Dennis}
\email[]{graham.dennis@anu.edu.au}
%\homepage[]{Your web page}
%\thanks{}
%\altaffiliation{}

\author{J.J. Hope}
\affiliation{Princeton Plasma Physics Laboratory, PO Box 451, Princeton, NJ 08543, USA}

% Collaboration name, if desired (requires use of superscriptaddress option in \documentclass). 
% \noaffiliation is required (may also be used with the \author command).
%\collaboration{}
%\noaffiliation

\date{\today}

\begin{abstract}
% insert abstract here
We derive a quasi-discrete Hankel transform (QDHT) appropriate for problems with Neumann boundary conditions.  Our method applies the ideas of \citet{Yu:1998} and \citet{Guizar-Sicairos:2004} which derive the QDHT assuming Dirichlet boundary conditions.

\end{abstract}

\pacs{}% insert suggested PACS numbers in braces on next line

\maketitle %\maketitle must follow title, authors, abstract and \pacs

% Body of paper goes here. Use proper sectioning commands. 
% References should be done using the \cite, \ref, and \label commands
\section{Introduction}
\label{sec:Introduction}

History: The original idea formulation of the zero-order quasi-discrete Hankel transform was developed by \citet{Yu:1998}.  This was generalized to arbitrary integer order transforms by \citet{Guizar-Sicairos:2004}.  The QDHT is not the only formulation of a discrete Hankel transform.  There are a number (see \citep{Yu:1998} for a list) of others, some of which use the DFT, but they all suffer problems with accuracy and stability.  Yu's formulation is slower ($O(N^2)$ time) than some of these `fast' Hankel transforms which can be performed in $O(N \log N)$ time, but the stability of the method is critical to applications where the DHT must be applied repeatedly, such as solving PDEs using spectral methods.

I haven't found anything particularly interesting in more recent literature, but I should probably read \citet{Cerjan:2007} before we publish to work out if it should be cited.

Of course we should cite the XMDS paper \citet{Dennis:2013}.

\section{Derivation of the method}
\label{sec:Derivation}



\bibliography{HankelNeumann.bib}


\end{document}
%
% ****** End of file aiptemplate.tex ******
