%% ****** Start of file aiptemplate.tex ****** %
%%
%%   This file is part of the files in the distribution of AIP substyles for REVTeX4.
%%   Version 4.1 of 9 October 2009.
%%
%
% This is a template for producing documents for use with 
% the REVTEX 4.1 document class and the AIP substyles.
% 
% Copy this file to another name and then work on that file.
% That way, you always have this original template file to use.

\documentclass[aip,amsmath,amssymb,reprint,twocolumn]{revtex4-1}
%\documentclass[aip,reprint]{revtex4-1}

\usepackage{graphicx,hyperref}
% \usepackage{pdfsync}

\newcommand{\relphantom}[1]{\phantom{\mathrel{#1}}}

\begin{document}

% Use the \preprint command to place your local institutional report number 
% on the title page in preprint mode.
% Multiple \preprint commands are allowed.
%\preprint{}

\title{Quasi-discrete Hankel transform with Neumann boundary conditions} %Title of paper

% repeat the \author .. \affiliation  etc. as needed
% \email, \thanks, \homepage, \altaffiliation all apply to the current author.
% Explanatory text should go in the []'s, 
% actual e-mail address or url should go in the {}'s for \email and \homepage.
% Please use the appropriate macro for the type of information

% \affiliation command applies to all authors since the last \affiliation command. 
% The \affiliation command should follow the other information.

\author{G.R. Dennis}
\email[]{graham.dennis@anu.edu.au}
%\homepage[]{Your web page}
%\thanks{}
%\altaffiliation{}

\author{J.J. Hope}
\affiliation{Research School of Physics and Engineering, Australian National University, ACT 0200, Australia}

% Collaboration name, if desired (requires use of superscriptaddress option in \documentclass). 
% \noaffiliation is required (may also be used with the \author command).
%\collaboration{}
%\noaffiliation

\date{\today}

\begin{abstract}
% insert abstract here
We derive a quasi-discrete Hankel transform (QDHT) appropriate for problems with Neumann boundary conditions.  Our method applies the ideas of \citet{Yu:1998} and \citet{Guizar-Sicairos:2004} which derive the QDHT assuming Dirichlet boundary conditions.

\end{abstract}

\pacs{}% insert suggested PACS numbers in braces on next line

\maketitle %\maketitle must follow title, authors, abstract and \pacs

% Body of paper goes here. Use proper sectioning commands. 
% References should be done using the \cite, \ref, and \label commands
\section{Introduction}
\label{sec:Introduction}

History: The original idea formulation of the zero-order quasi-discrete Hankel transform (QDHT) was developed by \citet{Yu:1998}.  This was generalized to arbitrary integer order transforms by \citet{Guizar-Sicairos:2004}.  The QDHT is not the only formulation of a discrete Hankel transform.  There are a number (see \citep{Yu:1998} for a list) of others, some of which use the DFT, but they all suffer problems with accuracy and stability.  Yu's formulation is slower ($O(N^2)$ time) than some of these `fast' Hankel transforms which can be performed in $O(N \log N)$ time, but the stability of the method is critical to applications where the DHT must be applied repeatedly, such as solving PDEs using spectral methods.

Other potentially relevant literature: \citet{Ronen:2006} applied the idea to Bogoliubov modes in a cylindrical trap which was how I found out about this, the method is widely applied in the optics literature (see the many citations of \citet{Guizar-Sicairos:2004}) where I fear the Dirichlet boundary condition is more useful.

I haven't found anything particularly interesting in more recent literature, but I should probably read \citet{Cerjan:2007} before we publish to work out if it should be cited.

Of course we should cite the XMDS paper \citep{Dennis:2013}.

One of the useful properties of the Hankel transform is that it provides accurate computation of the Laplacian for problems with radial symmetry:
\begin{align}
  \nabla^2 \left[f(r) e^{i m \theta}\right] &= \left\{\mathcal{H}^{-1}_m\left[(-k^2)\tilde{f}(k)\right](r)\right\}e^{i m \theta}
\end{align}
where the $m$-order Hankel transform and its inverse are defined by
\begin{align}
  \mathcal{H}_m[f](k) &= \tilde{f}(k) = \int_0^{\infty} r f(r) J_m(k r) \, dr \\
  \mathcal{H}^{-1}_m[\tilde{f}](r) &= f(r) = \int_0^{\infty} k \tilde{f}(k) J_m(k r)\, dk,
\end{align}
where $J_m(r)$ is the Bessel function of the first kind of order $m$. Or improve the notation as appropriate.  I've never actually seen the $\mathcal{H}_m$, $\mathcal{H}^{-1}_m$ notation used anywhere.

\section{Derivation of the method}
\label{sec:Derivation}

The definitions of the Hankel transform and its inverse are
\begin{align}
  f(r) &= \int_0^\infty k \tilde{f}(k) J_{m}(k r)\, dk, \\
  \tilde{f}(k) &= \int_0^\infty r f(r) J_{m}(k r)\, dr.
\end{align}
The Hankel transform is an orthogonal function decomposition which transforms a function $f(r)$ into an integral over the order-$m$ Bessel functions. The Bessel functions are orthogonal with respect to the weight factor $r$ and satisfy the orthogonality relation
\begin{align}
  \int_0^\infty r J_m(k r) J_m(k' r) \,dr &= \frac{\delta(k - k')}{k}.
\end{align}

Our goal is to derive a similar orthogonal function decomposition over the finite spatial domain $0 \leq r \leq R$ and the corresponding wavenumber domain $0 \leq k \leq K$.  The quasi-discrete Hankel transform and its inverse the quasi-discrete inverse Hankel transform (QDIHT) satisfy
\begin{align}
  f(r) &= \int_0^K k \tilde{f}(k) J_{m}(k r)\, dk,  \label{eq:QDHT}\\
  \tilde{f}(k) &= \int_0^R r f(r) J_{m}(k r)\, dr,  \label{eq:QDIHT}
\end{align}
where $f(r)$ and $\tilde{f}(k)$ are decomposed as a sum of Bessel functions
\begin{align}
  f(r) &= \sum_i f_i J_m(k_i r), & \tilde{f}(k) &= \sum_i \tilde{f}_i J_m(k r_i), \label{eq:Decomposition}
\end{align}
where the $r_i$ and $k_i$ are respectively the grid points of the spatial and wavenumber domains and their distribution has not yet been specified.

\begin{widetext}
For the basis functions in the decomposition given by Eq.~\eqref{eq:Decomposition} to be orthogonal, we require for all $i \neq j$
\begin{align}
  \int_0^{R} r J_m(k_i r) J_m(k_j r)\, dr &= \frac{R}{k_i^2-k_j^2} \left[k_j J_m(k_i R) J_m'(k_j R) - k_i J_m(k_j R) J_m'(k_i R)\right] = 0. \label{eq:DiscreteOrthogonality}
\end{align}
This discrete orthogonality relationship can be satisfied by choosing the $k_i$ such that either $J_m(k_i R) = 0$ or $J_m'(k_i R) = 0$.  In the former case $f(r)$ will satisfy Dirichlet boundary conditions at $r=R$ (i.e.\ $f(R) = 0$) and in the latter case $f(r)$ will satisfy Neumann boundary conditions (i.e.\ $f'(R) = 0$).  The case of Dirichlet boundary conditions has been considered before \citep{Yu:1998,Guizar-Sicairos:2004} and in this paper we focus on the Neumann boundary condition case.
\end{widetext}

For the $k_i$ to satisfy $J_m'(k_i R) = 0$, the $k_i$ must be given by
\begin{align}
  k_i &= j_{m,i}'/R,
\end{align}
where $j_{m,i}'$ is defined as in \citet{Abramowitz:1972} as the $i$th positive zero of the derivative of the Bessel function of order $m$, except that $r=0$ is counted as the first zero of $J_0(r)$.  The symmetry of Eqs.~\eqref{eq:QDHT} and \eqref{eq:QDIHT} implies that the spatial grid points $r_i$ are similarly given by
\begin{align}
  r_i &= j_{m,i}'/K.
\end{align}

Our basis functions now satisfy the orthogonality condition
\begin{align}
  \int_0^R r J_m(k_i r) J_m(k_j r)\, dr &= \frac{1}{2} \delta_{ij} \left(R^2 - \frac{m^2}{k_i^2}\right) J_m^2(k_i R)
\end{align}

Substituting the decomposition of $\tilde{f}(k)$ given by \eqref{eq:Decomposition} into the definition of the QDHT \eqref{eq:QDHT} we obtain
\begin{align}
  f(r_j) &= \int_0^K k \sum_i \tilde{f}_i J_m(k r_i) J_m(k r_j)\, dk, \\
  &= \frac{1}{2}\tilde{f}_j \left(K^2 - \frac{m^2}{r_j^2}\right) J_m^2(K r_j).
\end{align}
Next substituting the decomposition of $f(r)$ gives
\begin{align}
  \sum_i f_i J_m(k_i r_j) &= \frac{1}{2}\tilde{f}_j \left(K^2 - \frac{m^2}{r_j^2}\right) J_m^2(K r_j), \\
  \implies \tilde{f}_j &= \frac{2}{K^2 - m^2/r_j^2} \frac{1}{J_m^2(K r_j)} \sum_i f_i J_m(k_i r_j), \\
  \implies f_i &= \frac{2}{R^2 - m^2/k_i^2} \frac{1}{J_m^2(k_i R)} \sum_j \tilde{f}_j J_m(k_i r_j),
\end{align}
where the last line follows by symmetry between the QDHT and the QDIHT.

Finally
\begin{align}
  f(r_j) &= \sum_i \tilde{f}(k_i) \frac{2}{R^2 - m^2/k_i^2}\frac{1}{J_m^2(k_i R)} J_m(k_i r_j)
\end{align}


\section{Error analysis}
\label{sec:ErrorAnalysis}

\section{Example calculation}
\label{sec:Example}
Ideas for example: wave equation (should be similar to shallow water waves in a bucket); radial heat propagation in an isolated disk; whatever problem David Zwicker is solving.

\section{Conclusion}
\label{sec:Conclusion}



\bibliography{HankelNeumann.bib}


\end{document}
%
% ****** End of file aiptemplate.tex ******
